
\section{Grafik APIs}

\subsection{Was sind Grafik APIs?}
Eine Grafik API ist eine einheitliche Schnittstelle zur Programmierung einer Grafikkarte (GPU\footnote{Streng genommen sind eine Grafikkarte und eine GPU nicht dass selbe, viel mehr ist die GPU ein Teil einer Grafikkarte. Im Rahmen dieses Berichtes werden die beiden Begriffe jedoch als Synonyme genutzt.}). Dies ist notwendig, da verschiedene GPUs unterschiedlich Angesprochen werden und ohne eine solche Grafik API ein Programm für verschiedene GPUs unterschiedlich implementiert werden müsste \cite{samsung_what_is_a_graphics_api}. Visualisiert würde ein Programm also so aussehen:

\begin{tcolorbox}
    \begin{tcolorbox}
        \begin{center}
            High-Level Anwendung
        \end{center}
    \end{tcolorbox}
    \begin{tcolorbox}
        \begin{center}
            Grafik API
        \end{center}
    \end{tcolorbox}
    \begin{tcolorbox}
        \begin{center}
            GPU Treiber
        \end{center}
    \end{tcolorbox}
\end{tcolorbox}

Wie zu sehen ist, ist auf der obersten Schickt die Anwendung, die etwas grafisch Darstellen möchte. Diese gibt die jeweiligen Befehle dann weiter an die Grafik API, die die Befehle dann wiederrum an den jeweiligen GPU Treiber weiter gibt. Dabei ist zu beachten, dass die Anwendungs-Schickt in der Regel wieder in weitere, kleinere Schiten unterteilt werden kann - üblicherweise wird der Zugriff auf Grafik APIs durch Frameworks abstrahiert.

\subsection{Unterschiedliche APIs}
Auf dem Markt gibt es diverse Grafik APIs. Die in diesem Bericht behandelten APIs sind OpenGL (\url{https://www.opengl.org/}), Vukan (\url{https://www.khronos.org/vulkan/}) und Direct3D\footnote{Der Name DirectX wird oftmals an Stelle von Direct3D verwendet. Allerdings ist DirectX eine Sammlung von verschiedenen Bibliotheken (unter anderem Direct3D). Direct3D ist also die GrafikAPI, die in DirectX enthalten ist.} (\url{https://docs.microsoft.com/en-us/windows/desktop/direct3d}).

\subsection{Hardware}
Um zu verstehen warum Grafik APIs funktionieren wie sie funktionieren ist es notwendig zu verstehen wie die Hardware auf der sie laufen, die GPUs, funktionieren. Dabei ist eine GPU nicht allzu unterschiedlich von einer CPU, beide sind Recheneinheiten. Außerdem hat auch jede GPU einen Arbeitsspeicher (genannt Video-RAM oder auch VRAM). Der große Unterschied zwischen einer GPU und einer CPU besteht allerdings daraus, dass CPUs vergleichsweise hohe Taktraten und wenige Rechenkerne haben. GPUs hingegen haben deutlich geringere Taktraten, aber dafür deutlich mehr Kerne. Zum Vergleich: Ein moderner (Stand Anfang 2019) Intel I7 Prozessor hat acht Rechenkerne und eine Taktrate von 3,60 GHz \cite{intel_i7_9700k_processor}. Eine ebenfalls moderne Nvidia GTX 1080 Grafikkarte hat hingegen 2560 Rechenkerne und eine Taktrate von 1,607 GHz \cite{nvidia_gtx_1080_gpu}. Letztendlich können CPUs also nur wenige Dinge gleichzeitig berechnen, diese aber dafür sehr schnell. GPUs können sehr viele Dinge gleichzeitig berechnen, benötigen für die einzelnen Aufgaben jedoch länger.