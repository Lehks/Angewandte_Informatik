
%\section{Einführung}

%Im Folgenden werden einige Themen durch Codebeispiele verdeutlicht. Dieser Code ist immer in der Sprache C geschrieben\footnote{Es wird ausschließlich C, GLSL und HLSL verwendet (die beiden letzteren Sprachen sind spezifisch für die jeweiligen Grafik APIs, sie werden deshalb weiter unten erläutert). Zudem ist der C Code ist mit C++ kompatibel.}.

%Damit der Text so kurz wie möglich ist, werden nur die essenziellen Teile des Codes direkt im Text selbst eingefügt. Die kompletten, lauffähigen Programme existieren vollständig dokumentiert auf einem GitHub Repository (\url{https://github.com/Lehks/Angewandte_Informatik_Code}). Die Datei \texttt{\href{https://github.com/Lehks/Angewandte_Informatik_Code/blob/master/README.md}{README.md}} enthält weitere Informationen über den Inhalt und die Nutzung dieses Repositories.

%Codebeispiele werden immer in dem folgenden Format angegeben:
%\includeccode{example-1/main.c}{6}{10}
%Es ist zu Beachten, dass in der Regel nur ein Teil eines Quelltext Datei angegeben wird. Damit der Ausschnitt in der originalen Datei gefunden werden kann, wird aus diesem Grund zusätzlich immer der Dateiname angegeben\footnote{Dieser ist relativ zu dem Wurzelverzeichnis des oben genannten GitHub-Repositories.}. Zudem stehen links neben dem Code immer die Zeilennummern, die der Ausschnitt auch in der Quellcode Datei hat.
