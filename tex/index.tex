\documentclass[10pt,conference,a4paper]{IEEEtran}

\usepackage[utf8]{inputenc}
\usepackage[german]{babel}
\usepackage{hyperref}
\usepackage{listings}
\usepackage{color}
%\usepackage[style=numberic-comp,subentry]{biblatex}
\usepackage{apacite}
\usepackage{verbatim}
\usepackage{tcolorbox}
\usepackage{tikz}

\definecolor{code_bg}{rgb}{0.7,0.7,0.7}

% Source of command (last viewed 22.01.2019): https://tex.stackexchange.com/a/294990
\newcommand{\linksymbol}
{
    \tikz[x=1.2ex, y=1.2ex, baseline=-0.05ex]{
        \begin{scope}[x=1ex, y=1ex]
            \clip (-0.1,-0.1) 
                --++ (-0, 1.2) 
                --++ (0.6, 0) 
                --++ (0, -0.6) 
                --++ (0.6, 0) 
                --++ (0, -1);
            \path[draw, 
                line width = 0.5, 
                rounded corners=0.5] 
                (0,0) rectangle (1,1);
        \end{scope}
        \path[draw, line width = 0.5] (0.5, 0.5) 
            -- (1, 1);
        \path[draw, line width = 0.5] (0.6, 1) 
            -- (1, 1) -- (1, 0.6);
        }
}

\newcommand{\includeccode}[3]
{
    \begin{tcolorbox}[title=C/C++ \hfill \href{https://github.com/Lehks/Angewandte_Informatik_Code/blob/master/#1\#L#2}{#1\linksymbol},top=0pt,bottom=0pt]
        \lstinputlisting[style=c_style, firstline=#2, firstnumber=#2, lastline=#3]{code/#1}
    \end{tcolorbox}
}

\newcommand{\includeglslcode}[3]
{
    \begin{tcolorbox}[title=GLSL \hfill \href{https://github.com/Lehks/Angewandte_Informatik_Code/blob/master/#1\#L#2}{#1},top=0pt,bottom=0pt]
        \lstinputlisting[style=glsl_style, firstline=#2, firstnumber=#2, lastline=#3]{code/#1}
    \end{tcolorbox}
}

\newcommand{\includehlslcode}[3]
{
    \begin{tcolorbox}[title=HLSL \hfill \href{https://github.com/Lehks/Angewandte_Informatik_Code/blob/master/#1\#L#2}{#1},top=0pt,bottom=0pt]
        \lstinputlisting[style=hlsl_style, firstline=#2, firstnumber=#2, lastline=#3]{code/#1}
    \end{tcolorbox}
}

\lstdefinestyle{base_style}
{
    %backgroundcolor=\color{code_bg},
    %frame=single,
    numbers=left,
    numbersep=5pt,
    %captionpos=b,
    %framexleftmargin=1.5em,
    xleftmargin=1em
}

\lstdefinestyle{c_style}
{
    language=c,
    style=base_style
}

\lstdefinestyle{glsl_style}
{
    language=c,
    style=base_style
}

\lstdefinestyle{hlsl_style}
{
    language=c,
    style=base_style
}

\title{Eine Einführung in 3D Grafik APIs}
\author{Lukas Reichmann}
\date{\today}

\begin{document}

\maketitle

\begin{abstract}
    Dieser Artikel soll eine Grundlegende Einführung in die 3D Grafik Berechnung mit Hilfe von Diversen Low-Level Grafik APIs geben. Zudem sollen die behandelten Themen durch Codebeispiele weiter erläutert werden.
\end{abstract}


%\section{Einführung}

%Im Folgenden werden einige Themen durch Codebeispiele verdeutlicht. Dieser Code ist immer in der Sprache C geschrieben\footnote{Es wird ausschließlich C, GLSL und HLSL verwendet (die beiden letzteren Sprachen sind spezifisch für die jeweiligen Grafik APIs, sie werden deshalb weiter unten erläutert). Zudem ist der C Code ist mit C++ kompatibel.}.

%Damit der Text so kurz wie möglich ist, werden nur die essenziellen Teile des Codes direkt im Text selbst eingefügt. Die kompletten, lauffähigen Programme existieren vollständig dokumentiert auf einem GitHub Repository (\url{https://github.com/Lehks/Angewandte_Informatik_Code}). Die Datei \texttt{\href{https://github.com/Lehks/Angewandte_Informatik_Code/blob/master/README.md}{README.md}} enthält weitere Informationen über den Inhalt und die Nutzung dieses Repositories.

%Codebeispiele werden immer in dem folgenden Format angegeben:
%\includeccode{example-1/main.c}{6}{10}
%Es ist zu Beachten, dass in der Regel nur ein Teil eines Quelltext Datei angegeben wird. Damit der Ausschnitt in der originalen Datei gefunden werden kann, wird aus diesem Grund zusätzlich immer der Dateiname angegeben\footnote{Dieser ist relativ zu dem Wurzelverzeichnis des oben genannten GitHub-Repositories.}. Zudem stehen links neben dem Code immer die Zeilennummern, die der Ausschnitt auch in der Quellcode Datei hat.


\section{Grafik APIs}

\subsection{Was sind Grafik APIs?}
Eine Grafik API ist eine einheitliche Schnittstelle zur Programmierung einer Grafikkarte (GPU\footnote{Streng genommen sind eine Grafikkarte und eine GPU nicht dass selbe, viel mehr ist die GPU ein Teil einer Grafikkarte. Im Rahmen dieses Berichtes werden die beiden Begriffe jedoch als Synonyme genutzt.}). Dies ist notwendig, da verschiedene GPUs unterschiedlich Angesprochen werden und ohne eine solche Grafik API ein Programm für verschiedene GPUs unterschiedlich implementiert werden müsste \cite{samsung_what_is_a_graphics_api}. Visualisiert würde ein Programm also so aussehen:

\begin{tcolorbox}
    \begin{tcolorbox}
        \begin{center}
            High-Level Anwendung
        \end{center}
    \end{tcolorbox}
    \begin{tcolorbox}
        \begin{center}
            Grafik API
        \end{center}
    \end{tcolorbox}
    \begin{tcolorbox}
        \begin{center}
            GPU Treiber
        \end{center}
    \end{tcolorbox}
\end{tcolorbox}

Wie zu sehen ist, ist auf der obersten Schickt die Anwendung, die etwas grafisch Darstellen möchte. Diese gibt die jeweiligen Befehle dann weiter an die Grafik API, die die Befehle dann wiederrum an den jeweiligen GPU Treiber weiter gibt. Dabei ist zu beachten, dass die Anwendungs-Schickt in der Regel wieder in weitere, kleinere Schiten unterteilt werden kann - üblicherweise wird der Zugriff auf Grafik APIs durch Frameworks abstrahiert.

\subsection{Unterschiedliche APIs}
Auf dem Markt gibt es diverse Grafik APIs. Die in diesem Bericht behandelten APIs sind OpenGL (\url{https://www.opengl.org/}), Vukan (\url{https://www.khronos.org/vulkan/}) und Direct3D\footnote{Der Name DirectX wird oftmals an stelle von Direct3D verwendet. Allerdings ist DirectX eine Sammlung von verschiedenen Bibliotheken (unter anderem Direct3D). Direct3D ist also die GrafikAPI, die in DirectX enthalten ist.} (\url{https://docs.microsoft.com/en-us/windows/desktop/direct3d}).

\subsection{Hardware}
Um zu verstehen warum Grafik APIs funktionieren wie sie funktionieren ist es notwendig zu verstehen wie die Hardware auf der sie laufen, die GPUs, funktionieren. Dabei ist eine GPU nicht allzu unterschiedlich von einer CPU, beide sind Recheneinheiten. Außerdem hat auch jede GPU einen Arbeitsspeicher (genannt Video-RAM oder auch VRAM). Der große Unterschied zwischen einer GPU und einer CPU besteht allerdings daraus, dass CPUs vergleichsweise hohe Taktraten und wenige Rechenkerne haben. GPUs hingegen haben deutlich geringere Taktraten, aber dafür deutlich mehr Kerne. Zum Vergleich: Ein moderner (Stand Anfang 2019) Intel I7 Prozessor hat acht Rechenkerne und eine Taktrate von 3,60 GHz \cite{intel_i7_9700k_processor}. Eine ebenfalls moderne Nvidia GTX 1080 Grafikkarte hat hingegen 2560 Rechenkerne und eine Taktrate von 1,607 GHz \cite{nvidia_gtx_1080_gpu}. Unterm Strich können CPUs also nur wenige Dinge gleichzeitig berechnen, diese aber dafür sehr schnell. GPUs können sehr viele Dinge gleichzeitig berechnen, benötigen für die einzelnen Aufgaben jedoch länger.

\section{Shader Pipeline}

Der String muss immernoch gefixt werden \cite{dfranz:book:2}.

\section{Buffer}

Tests müssen auch noch geschrieben werden \cite{dfranz:book:3}.

\section{Uniforms}

\bibliographystyle{apacite}
\bibliography{References}

\end{document}